Week 8:

Monday 15th March (11am)
Introduction to MS4024
Introduction to R
Notice of Assignment 1 (Brief description of package)
Monday 15th March (3pm)
Vectors
Thursday 18th March (3pm)
Exercise 1 submission
	                
 	                   
Week 9: 

Monday 22nd March (11am)
Writing functions
Monday 22nd March (3pm)
Basic graphics - barcharts, histograms and scatterplots
Thursday 25th March  (3pm)
Assignment 1 - submission (5%)
Assignment 2 - open book, in class assessment (10%)

Textbooks

Introductory Statistics with R. 
Peter Dalgaard. (Springer, 2002. ISBN 0-387-95475-9. 519.5/DAL)

A First course in statistical programming with R. 
W. John Braun and Duncan J. Murdoch. 
(Cambridge University Press, 2007. ISBN 978-0-521-69424-7.)

Assessment
This part of the module is marked using continuous assessment, with no end of semester exam. 
There are four assessments4 assessments weighted as 5%, 10%, 15% and 20%.

Downloading and Installing R




R can be started in the usual way by double-clicking on the R icon on the desktop.

Working Directory

R works best if you have a dedicated folder for each separate project - called the working directory.
 Create the directory/folder that will be used as the working folder, e.g. create a folder on your desktop titled Your_name
by right-clicking, then clicking New > Folder.
  Right-click on an existing R icon and click Copy.
  In the working folder, right-click and click Paste. 
The R icon will appear in the folder.


 

	     
Part 1 Introduction to R

R is an open source implementation of the S-Plus language. It is freely available and is licensed under the GPL. The language simplifies many statistical computations and can be a powerful tool. R is typically used by statisticians as a tool for direct data analysis.


R software and packages can be downloaded from Comprehensive R Archive Network (CRAN; http: //cran.r-project.org).
 
R is under very active development and is constantly gaining “market shares” in many fields related to statistics, such as finance, data mining, genomics and pharmaceuticals.
 
.

Part 2 Downloading and Installing R

Click on
Windows > base > R-version-win32.exe > Run
and follow the instructions to install the programme.


Part 4 Vectors and assignment

R operates on named data structures. The simplest such structure is the numeric vector, which is a single entity consisting of an ordered collection of numbers. To set up a vector
named x, say, consisting of five numbers, namely 10.4, 5.6, 3.1, 6.4 and 21.7, use the R command
> x <- c(10.4, 5.6, 3.1, 6.4, 21.7)

This is an assignment statement using the function c()

Part 5 Using the Help command

R has a built-in help facility. To get more information on any specific function, e.g. sqrt(), the command is
> help(sqrt)

An alternative is
> ?sqrt

We can also obtain help on features specified by special characters. In this case q
> help("[[")

Help is also available in HTML format by running
> help.start()

For more information use
> ?help



Part 6 Exercise 1




Useful links:

http://cran.r-project.org/doc/manuals/R-intro.pdf
 
 
 
All submissions are to be pdf files.
Familiarity with Winedt is assumed.
Description: A plug-in for using WinEdt as an editor for R
R-Winedt is a graphical user interface for R
Installing the package "R winedt"
To install RWinEdt:
In the R enviroment, click on 'packages' on the menu bar.
Select 'install packages(s) from local zip files'.
Navigate to the working directory and select 'RWinEdt.zip'.
RWinEdt will now be added to the list of available packages.
To open RWinEdt:
click on 'packages' on the menu bar again,
clink on 'load packages'.
Select 'RWinEdt'.
RWinEdt is now installed on your desktop, and is accessible through the 'start menu'.
You can now write and edit your R code.
To compile R code, select a code segment, and then click on the 'R-paste' button
on the menubar. The R environment console must always be active.
R file are saved with '.R' extension.
Open 'Week8.R' and try running some of the code segments.
R code can also be edited using other GUIs such as tinn-r, or even notepad.
Notepad can also be used to edit code until you are fully familiarised with RWinEdt.

